\documentclass[a4paper,landscape]{article}
\RequirePackage{tikz}
\RequirePackage[overlay]{textpos}
\setlength{\TPHorizModule}{1cm}
\setlength{\TPVertModule}{1cm}

\usepackage{unicode}

%%%%%%%%%
% Fonts %
%%%%%%%%%
\RequirePackage[quiet]{fontspec}

\newfontfamily\bodyfont[]{HelveticaNeue LT 53 Ex}
\newfontfamily\thinfont[]{仿宋}
\newfontfamily\headingfont[]{黑体}
\newfontfamily\normalfont[]{宋体}

\defaultfontfeatures{Mapping=tex-text}
\setmainfont[Mapping=tex-text, Color=textcolor]{宋体}


%%%%%%%%%%
% Colors %
%%%%%%%%%%

\RequirePackage{xcolor}

\definecolor{white}{RGB}{255,255,255}

\definecolor{darkgray}{HTML}{333333}
\definecolor{gray}{HTML}{4D4D4D}
\definecolor{lightgray}{HTML}{999999}

\definecolor{green}{HTML}{C2E15F}
\definecolor{orange}{HTML}{FDA333}
\definecolor{purple}{HTML}{D3A4F9}
\definecolor{red}{HTML}{FB4485}
\definecolor{blue}{HTML}{6CE0F1}

\colorlet{fillheader}{gray}
\colorlet{header}{white}
\colorlet{textcolor}{gray}
\colorlet{headercolor}{gray}


%%%%%%%%%%%%%
% Structure %
%%%%%%%%%%%%%
\RequirePackage{parskip}

\newcounter{colorCounter}
\def\@sectioncolor#1#2{%
  {%
    \color{%
      \ifcase\value{colorCounter}%
        blue\or%
        red\or%
        orange\or%
        green\or%
        purple\or%
        brown\else%
        headercolor\fi%
    } #1#2%
  }%
  \stepcounter{colorCounter}%
}

\renewcommand{\section}[1]{
  \par\vspace{\baselineskip}
  {%
    \LARGE\headingfont\color{headercolor}%
    \@sectioncolor #1%
  }
  \par\vspace{\baselineskip}
}

\renewcommand{\subsection}[2]{
  \par\vspace{.5\baselineskip}%
  \Large\headingfont\color{headercolor} #2%
  \par\vspace{.25\baselineskip}%
}

\pagestyle{empty}

%%%%%%%%%%%%%%%%%%%%
% List environment %
%%%%%%%%%%%%%%%%%%%%

\setlength{\tabcolsep}{0pt}
\newenvironment{entrylist}{%
  \begin{tabular*}{\textwidth}{@{\extracolsep{\fill}}ll}
}{%
  \end{tabular*}
}
\renewcommand{\bfseries}{\headingfont\color{gray}}
\newcommand{\entry}[4]{%
  #1&\parbox[t]{16cm}{%
    \textbf{#2}%
    \hfill%
    {\footnotesize\addfontfeature{Color=lightgray} #3}\\%
    \normalfont#4\vspace{\parsep}%
  }\\}
  
\newcommand{\notitleentry}[2]{%
  #1&\parbox[t]{16cm}{%
    \normalfont#2%
    \hfill\\%
  }\\}


\usepackage[top=0in, bottom=0in, left=0in, right=1.5cm, nohead,nofoot]{geometry}

\begin{document}
\setlength{\fboxsep}{0pt}
 \colorbox{gray}{
  \begin{minipage}[t][\paperheight]{0.2\paperwidth}
    \begin{flushright}
        \fontsize{40pt}{72pt}\color{gray}{\_\_\_}% use this to keep the two minipage at the same top height
      \\
        \begin{textblock}{5.5}(0.2, 3)
        \begin{flushright}
            \fontsize{40pt}{72pt}\color{white}%
              {\bodyfont Soul}
             \vspace{0.3cm}\\
             \fontsize{14pt}{24pt}\color{white}%
              {\thinfont 赵 翔}
              \vspace{0.3cm}\\
             \fontsize{10pt}{12pt}\color{white}%
              {\thinfont 软件工程师}%
         \end{flushright}
        \end{textblock}

        \begin{textblock}{5.5}(0.2, 15)
        \begin{flushright}
            \let\oldsection\section
            \renewcommand{\section}[1]{
                    \par\vspace{\baselineskip}{\Large\headingfont\color{white} \hfill#1}
            }
            
            \section{联系方式}
            \\
            {\fontsize{10pt}{12pt}\color{white}+86 151 0613 7119}
		 \\
            {\fontsize{10pt}{12pt}\color{white}soulzx2010@gmail.com}
		 \\
		 {\fontsize{10pt}{12pt}\color{white}github.com/soulzhao}
            \\
            {\fontsize{10pt}{12pt}\color{white}苏州市工业园区}
		  ~
            \let\section\oldsection
         \end{flushright}
        \end{textblock}
    \end{flushright}
\end{minipage}
}
%
 \colorbox{white}{
\begin{minipage}[t][\paperheight]{0.7\paperwidth}
    \begin{flushleft}

%----------------------------------------------------------------------------------------
%   EDUCATION SECTION
%----------------------------------------------------------------------------------------

\section{教育经历}

\begin{entrylist}
%------------------------------------------------
\entry
{2008 -- 2012}
{计算机科学技术学士学位}
{苏州科技大学}
{本科期间专业前3, GPA 3.7, 优秀毕业生杰出代表}
%------------------------------------------------
\end{entrylist}

%----------------------------------------------------------------------------------------
%   CAREER SECTION
%----------------------------------------------------------------------------------------

\section{职业生涯}

\begin{entrylist}
%------------------------------------------------
\entry
{2012 -- Now}
{甲骨文(中国)软件系统有限公司}
{工业园区, 苏州, 中国}
{\emph{2年 融合人力资源管理应用工程师} \\
负责甲骨文融合应用 -- 人力资源系统的开发工作。我主要负责成本模块的维护与开发工作,也参与商务\\智能报表模块的维护和开发。使用Java语言工作。\\}
%------------------------------------------------
\entry
{2011 -- 2012}
{旺宏微电子(苏州)有限公司 (实习)}
{工业园区, 苏州, 中国}
{\emph{助理软件工程师} \\
作为助理软件工程师在此公司做实习工作。主要负责安卓系统的优化和Linux驱动开发。我当时主要负责\\的是基于安卓2.3系统平台的传感器模块的维护和开发工作。\\}
%------------------------------------------------
\end{entrylist}

%----------------------------------------------------------------------------------------
%   PROJECT EXPERIENCE SECTION
%----------------------------------------------------------------------------------------

\section{项目经验}

\begin{entrylist}
%------------------------------------------------
\entry
{2014 -- Now}
{电动汽车信息采集及分析系统}
{中国科学院信息工程研究所}
{与中科院的一位博士后合作,做为外聘专家, 负责网络相关信息的采集和分析。\\使用Python开发了一个多线程爬虫去抓取数以百计的相关网站上的特定信息。实现了一个简易线程池,\\采用XML模板定制的方法为每一个网站定制抓取内容,方便用户使用。\\}
%------------------------------------------------
\entry
{2011 -- 2012}
{城市应急道路维护及选择系统}
{上海交通大学}
{与该校交通学院的一位博士展开合作,共同完成这个工程。这是一个基于.NET平台的,用C\#写成的系统,\\实现了众多的算法。
这个工具肯定根据路况信息来计算在危险情况下的道路可行性,以及为抢险队智\\能的去寻找最优路径(时间上,空间上),为物资分发提供最优方案的系统。\\
实现的算法包括: Dijistra算法, 蒙特卡洛算法, 遗传算法以及其他在图论中经常使用的算法。\\}
%------------------------------------------------
\end{entrylist}

%----------------------------------------------------------------------------------------
%	INTERESTS SECTION
%----------------------------------------------------------------------------------------

\section{自我陈述}
\begin{entrylist}

\notitleentry
{编码能力:}
{熟练使用Java, C\#, C/C++, Matlab, Python等语言,了解js, php, bash等。\\
熟悉多线程编程,熟悉设计模式。算法方面,我自认为是一个能快速学习并实现算法的人。\\}
\notitleentry
{交流能力:}
{良好的中英语交流能力。我可以清楚明白的用英文表述自己想法,不论是书面交流还是当面交流。在\\Oracle,我的交流能力给我的老板留下了深刻印象,尤其是书面交流能力,因为我曾是我所在大学的\\征文比赛冠军。\\}
\notitleentry
{自我评价:}
{我认为我是一个能快速学习,并且能快速上手各种工作的人,我能通过各种方面解决工作中的问题。\\同时,我也是一个负责任的,专注的人。\\}


\end{entrylist}

    \end{flushleft}
\end{minipage}
}
\end{document}